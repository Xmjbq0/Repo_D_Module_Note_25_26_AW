% Week 1 - Recall of smooth affine AG over C
% Local macros for common symbols
\providecommand{\C}{\mathbb{C}}
\providecommand{\Spec}{\operatorname{Spec}}
\providecommand{\T}{\mathcal{T}}
\providecommand{\A}{\mathbb{A}}
\providecommand{\Sym}{\operatorname{Sym}}
\providecommand{\End}{\operatorname{End}}
\providecommand{\D}{\mathcal{D}}
\providecommand{\Hom}{\operatorname{Hom}}
\providecommand{\Der}{\operatorname{Der}}
\providecommand{\gr}{\operatorname{gr}}
\providecommand{\ord}{\operatorname{ord}}
\providecommand{\Mod}{\mathrm{Mod}}

% 为了方便,定义一些快捷命令
\newcommand{\Dx}{\mathcal{D}_X}
% \newcommand{\Mod}{\operatorname{Mod}}
\newcommand{\ox}{\omega_X}
\newcommand{\oxinv}{\omega_X^{-1}}
\newcommand{\oxstar}{\omega_X^*}

\section{Recall of smooth affine algebraic geometry over $\C$}
Let $R$ be an integral commutative $\C$-algebra of finite type. Set
\[
  X := \Spec R,
\]
 the affine algebraic variety associated to $R$, endowed with the Zariski topology.
For an ideal $I\subseteq R$, write
\[
  V(I) := \{\, \mathfrak p\in \Spec R \mid \mathfrak p\supseteq I \,\},
\]
so that a subset $V\subseteq X$ is closed iff $V=V(I)$ for some $I$.

We denote the module of K\"ahler differentials of $R$ over $\C$ by
\[
  \Omega_R = \Omega_X = \Omega_{R/\C}.
\]
\begin{definition}[K\"ahler differentials]\label{def:kahler}
The $R$-module $\Omega_{R/\C}$ is generated by formal symbols $\mathrm df$ with $f\in R$,
subject to the relations, for all $f,g\in R$ and $a,b\in\C$,
\begin{align*}
  \mathrm d(fg) &= f\,\mathrm dg + g\,\mathrm df &&\text{(Leibniz rule)},\\
  \mathrm d(af+bg) &= a\,\mathrm df + b\,\mathrm dg.
\end{align*}
\end{definition}
\begin{exercise}
Show that $\mathrm dc=0$ for every constant $c\in\C$.
\end{exercise}
\begin{proof}[Solution to the exercise]
Using linearity and the Leibniz rule,
\(\mathrm d(1)=\mathrm d(1\cdot 1)=1\,\mathrm d1+1\,\mathrm d1=2\,\mathrm d1\), hence
\(\mathrm d1=0\). For $c\in\C$ we have
\(\mathrm dc=\mathrm d(c\cdot 1)=c\,\mathrm d1+1\,\mathrm dc=\mathrm dc\), and the only possibility is
\(\mathrm dc=0\).
\end{proof}

Define the $R$-module
\[
  \T_X := \Hom_R\!\big(\Omega_{R/\C},\, R\big) \cong \Der_{\C}(R,R),
\]
called the module of $\C$-derivations on $X$ (the vector fields on $X$).

\begin{definition}[Smoothness]
We say that $R$ (or $X=\Spec R$) is \emph{smooth} if $\Omega_{R/\C}$ is a free $R$-module of
finite rank. In this case, the rank
\[
  n := \operatorname{rank}_R(\Omega_{R/\C})
\]
 is the dimension of $X$.
\end{definition}

If $X$ is smooth, a tuple $(x_1,\dots,x_n)$ with $x_i\in R$ is called an \emph{algebraic
coordinate system} if
\[
  \Omega_{R/\C} \cong \bigoplus_{i=1}^n R\,\mathrm d x_i.
\]
Equivalently, the dual basis gives
\[
  \T_X \cong \bigoplus_{i=1}^n R\,\partial_{x_i},
\]
where $\partial_{x_i}$ are the vector fields dual to $\mathrm d x_i$.

\begin{example}[Polynomial ring]
Let $R=\C[x_1,\dots,x_n]$. Then
\[
  \Omega_{R/\C} \cong \bigoplus_{i=1}^n R\,\mathrm d x_i,
\]
so $X=\Spec R\simeq \A^n_{\C}$ is smooth and $(x_1,\dots,x_n)$ is an algebraic coordinate system.
\end{example}

From now on assume $X=\Spec R$ is smooth affine and $(x_1,\dots,x_n)$ are algebraic coordinates.
The symmetric algebra of vector fields is
\[
  \Sym^{\bullet}\T_X = \bigoplus_{m\ge 0}\Sym^{m}\T_X \simeq R[\xi_1,\dots,\xi_n],
\]
canonically an $R$-algebra (identify $\xi_i$ with the class of $\partial_{x_i}$). Define the
cotangent bundle
\[
  T^{\ast}X := \Spec\, \Sym^{\bullet}\T_X \xrightarrow{\;\pi\;} X.
\]
and $\pi$ is induced by the $R$-algebra map $R = \Sym^{0}\T_X \subset \bigoplus_{m\ge 0}\Sym^{m}\T_X$.

\section{D-modules on smooth affine varieties}
Let $\End_{\C}(R)$ denote the $\C$-algebra of all $\C$-linear maps $R\to R$. There are natural
embeddings
\[
  R\hookrightarrow \End_{\C}(R),\quad f\longmapsto (r\mapsto fr),
\]
and
\[
  \T_X\hookrightarrow \End_{\C}(R),\quad v\longmapsto v':=[f\mapsto v(\mathrm d f)],
  \begin{tikzcd}[row sep=3em, column sep=1.5em]
      R \arrow[rr, "d"] \arrow[dr, "v'"]  % v' 在箭头上方(默认)
      & & \Omega_R \arrow[dl, "v"]       % v 在箭头下方(不使用 ' 换边)
      \\
      & R &
  \end{tikzcd}.
\]

\begin{definition}[Ring of differential operators]
The ring $\D_X$ of algebraic differential operators on $X$ is the $\C$-subalgebra of
$\End_{\C}(R)$ generated by $R$ and $\T_X$.
\end{definition}

\begin{example}[Weyl algebra]
If $X=\A^n_{\C}=\Spec \C[x_1,\dots,x_n]$, then
\[
  \D_X \cong \C[x_1,\dots,x_n]\langle \partial_{x_1},\dots,\partial_{x_n}\rangle,
\]
with relations
\[
  [\partial_{x_i},x_j]=\delta_{ij},\quad [x_i,x_j]=0,\quad [\partial_{x_i},\partial_{x_j}]=0.
\]
\end{example}

Thus $\D_X$ is non-commutative. Every operator $P\in\D_X$ can be written in multi-index notation as
\[
  P=\sum_{\alpha\in\mathbb N^n} f_{\alpha}\,\partial^{\alpha},\qquad
  \partial^{\alpha}:=\partial_{x_1}^{\alpha_1}\cdots\partial_{x_n}^{\alpha_n},\quad f_{\alpha}\in R.
\]

\paragraph{Basic modules.}\mbox{}\par

\noindent \ding{172} \quad The coordinate ring $R$ is a left $\D_X$-module where$\partial_{x_i}\cdot r = \partial_{x_i}(\mathrm d r)\in R$.
\noindent Equivalently, writing $\mathrm dr=\sum_i (\partial_{x_i}r)\,\mathrm d x_i$, we have $\partial_{x_i}r$ the
$i$-th coefficient.

\par\noindent \ding{173} \quad The top exterior power
\[
  \omega_X := \bigwedge{}^{n}\Omega_{R/\C} \cong R\,\mathrm d x_1\wedge\cdots\wedge\mathrm d x_n
\]
 is the dualizing module of $R$. It carries a natural right $\D_X$-module structure via the transpose
anti-involution $t\colon \D_X\to\D_X$ determined by
\[
  \prescript{t}{}{x_i}=x_i,\qquad \prescript{t}{}{\partial_{x_i}}=-\partial_{x_i}.
\]
If $P=\sum_{\alpha} f_{\alpha}\,\partial^{\alpha}$, then
\[
  \prescript{t}{}{P}=\sum_{\alpha} (-\partial)^{\alpha} f_{\alpha},
\]
and the action is
\[
  (r\,\mathrm d x_1\wedge\cdots\wedge\mathrm d x_n)\cdot P := \prescript{t}{}{P}\,r\,\mathrm d x_1\wedge\cdots\wedge\mathrm d x_n.
\]

\subsection*{Side change and opposite ring}
Let $A$ be a ring and let $A^{\mathrm{op}}$ denote the opposite ring. Then right $A$-modules are the
same as left $A^{\mathrm{op}}$-modules.

\begin{lemma}\label{lem:DXop}
There is a natural isomorphism of rings
\[
  \D_X^{\mathrm{op}} \simeq \omega_X \otimes_R \D_X \otimes_R \omega_X^{\!*},
\]
where $\omega_X^{\!*}:=\Hom_R(\omega_X,R) \cong R\,(\mathrm d x_1\wedge\cdots\wedge\mathrm d x_n)^{-1}$.
\end{lemma}
\begin{proof}
Define
\[
  \Phi(P^{\mathrm{op}}):=\mathrm d x_1\wedge\cdots\wedge\mathrm d x_n\;\otimes\; \prescript{t}{}{P}\;\otimes\;
  (\mathrm d x_1\wedge\cdots\wedge\mathrm d x_n)^{-1}.
\]
Then for $P,Q\in\D_X$ we have
\begin{align*}
  \Phi(P^{\mathrm{op}}Q^{\mathrm{op}})
  &=\Phi((QP)^{\mathrm{op}})\\
  &=\mathrm d x\,\otimes \prescript{t}{}{(QP)}\otimes (\mathrm d x)^{-1}\\
  &=\mathrm d x\,\otimes \prescript{t}{}{P}\prescript{t}{}{Q}\otimes (\mathrm d x)^{-1}\\
  &=\Phi(P^{\mathrm{op}})\,\Phi(Q^{\mathrm{op}}),
\end{align*}

so $\Phi$ is a ring homomorphism. It is bijective with inverse obtained by
contracting the outer factors and applying $t$ again. Compatibility with the
$R$-actions follows from the standard left/right $\D_X$-module structures on
$\omega_X$ and $\omega_X^{\!*}$. Hence $\Phi$ is an isomorphism of rings.
\end{proof}

Write $\Mod(\D_X^{\mathrm{op}})$ for the category of right $\D_X$-modules and $\Mod(\D_X)$ for the
category of left $\D_X$-modules.

\begin{theorem}\label{thm:left-right-equivalence}
Tensoring with the canonical bundle yields an equivalence of categories
\[
  \Mod(\D_X) \xrightarrow{\;\sim\;} \Mod(\D_X^{\mathrm{op}}),\qquad M\longmapsto \omega_X\otimes_R M,
\]
with quasi-inverse $N\longmapsto \omega_X^{\!*}\otimes_R N$.
\end{theorem}
\begin{proof}
    % 第一部分
    \noindent \ding{172} Let $M \in \Mod(\Dx)$.
    \begin{align*}
        \ox \otimes_R \Dx \otimes \oxinv \cdot (\ox \otimes M) 
        &= \ox \otimes (\Dx \cdot M) \\
        &\subseteq \ox \otimes M
    \end{align*}
    \[
        \implies \ox \otimes M \in \Mod(\Dx^{\mathrm{op}})
    \]

    % 第二部分
    \noindent \ding{173} Let $N \in \Mod(\Dx^{\mathrm{op}}) = \Mod(\ox \otimes_R \Dx \otimes \oxstar)$.
    \begin{align*}
        \Dx \cdot (\oxstar \otimes_R N) 
        &= (\oxstar \otimes \ox \otimes \Dx \otimes \oxstar) \cdot N \\
        &= \oxstar \otimes (\ox \otimes \Dx \otimes \oxstar \cdot N) \\
        &\subseteq \oxstar \otimes N
    \end{align*}
    \[
        \implies \oxstar \otimes_R N \in \Mod(\Dx) 
    \]
\end{proof}

In particular, we will henceforth only consider left $\D_X$-modules, written simply as $\D_X$-modules.

\section{Filtered rings I}
Let $A$ be a unital ring. A family $\{\Gamma_i A\}_{i\in\mathbb Z}$ is a $\mathbb Z$-filtration of $A$ if
\begin{itemize}
  \item $\Gamma_i A \subseteq A$;
  \item $\Gamma_i A \subseteq \Gamma_{i+1} A$ and $1\in \Gamma_0 A$;
  \item $\bigcup_{i\in\mathbb Z} \Gamma_i A = A$;
  \item $\Gamma_i A\cdot \Gamma_j A \subseteq \Gamma_{i+j} A$ for all $i,j$; and often we assume the
         filtration is separated: $\bigcap_{i\in\mathbb Z}\Gamma_i A=0$.
\end{itemize}
We say $A$ is $\mathbb Z_{\ge 0}$-filtered if additionally $\Gamma_i A=0$ for all $i<0$.

The associated graded ring is
\[
  \operatorname{gr}A = \bigoplus_{i\in\mathbb Z} \Gamma_i A/\Gamma_{i-1} A.
\]
The Rees ring (with respect to $\Gamma$) is
\[
  R^{\Gamma}(A) = \bigoplus_{i\in\mathbb Z} \Gamma_i A\,u^i \subseteq A[u,u^{-1}],
\]
which records the filtration multiplicatively.

\subsection*{Basic consequences}
\begin{lemma}\label{lem:rees-quotients}
For the Rees ring $R^{\Gamma}(A)$ one has natural isomorphisms
\[
  \gr A \simeq R^{\Gamma}(A)/u\,R^{\Gamma}(A),\qquad
  A \simeq R^{\Gamma}(A)/(u-1)\,R^{\Gamma}(A).
\]
\end{lemma}

\begin{proposition}\label{prop:noeth}
Let $A$ be a filtered ring with filtration $\Gamma$.
\begin{enumerate}
  \item If $R^{\Gamma}(A)$ is Noetherian, then both $A$ and $\gr A$ are Noetherian.
  \item If the filtration is nonnegative (i.e. $\Gamma_iA=0$ for $i<0$) and
        $\gr A$ is Noetherian, then $A$ is Noetherian.
\end{enumerate}
\end{proposition}
\begin{proof}
(1) By Lemma~\ref{lem:rees-quotients}, $\gr A$ and $A$ are homomorphic images of
the Rees ring $R^{\Gamma}(A)$. If the latter is Noetherian then so are its
quotients $\gr A$ and $A$.

(2) Assume the filtration is nonnegative, i.e. $\Gamma_iA=0$ for $i<0$ and
$A=\bigcup_{i\ge0}\Gamma_iA$. Suppose $I_1\subseteq I_2\subseteq\cdots$ is an
ascending chain of left ideals of $A$. Put graded ideals
\(\mathrm{gr}(I_j):=\bigoplus_{p\ge0}(I_j\cap \Gamma_pA+\Gamma_{p-1}A)/\Gamma_{p-1}A\)
in $\gr A$. Then $\mathrm{gr}(I_j)\subseteq \mathrm{gr}(I_{j+1})$ and the chain stabilizes because
$\gr A$ is Noetherian. Choose $N$ so that $\mathrm{gr}(I_j)=\mathrm{gr}(I_N)$ for all $j\ge N$.
We claim $I_j=I_N$ for all $j\ge N$, hence $A$ is Noetherian. Indeed, fix $j\ge N$ and
argue by induction on $p$ that $I_j\cap\Gamma_pA=I_N\cap\Gamma_pA$ for all $p$:
for $p=0$ it follows from equality of degree-$0$ pieces. If true for $p-1$, then
the equality of degree-$p$ pieces gives
\((I_j\cap \Gamma_pA+\Gamma_{p-1}A)/(\Gamma_{p-1}A)
 =(I_N\cap \Gamma_pA+\Gamma_{p-1}A)/(\Gamma_{p-1}A)\), hence
$I_j\cap\Gamma_pA=I_N\cap\Gamma_pA$. Taking unions over $p$ yields $I_j=I_N$.
Thus every ascending chain stabilizes and $A$ is Noetherian.
\end{proof}

\subsection*{Filtered modules and good filtrations}
Let $A$ be a filtered ring with filtration $\Gamma_\bullet A$ and let $M$ be a left
$A$-module.

\begin{definition}\label{def:filtered-module}
A $\mathbb Z$-filtration $\Omega_\bullet M$ is a family of submodules
$\{\Omega_i M\}_{i\in\mathbb Z}$ such that $\bigcup_i \Omega_iM=M$ and
\[
  \Gamma_iA\cdot \Omega_jM \subseteq \Omega_{i+j}M \quad \text{for all } i,j\in\mathbb Z.
\]
It is called \emph{good} if it is compatible and the Rees module
\[
  R^{\Omega}(M):=\bigoplus_{i\in\mathbb Z}\Omega_iM\,u^i
\]
is finitely generated over $R^{\Gamma}(A)$ (equivalently,
$\gr^{\Omega}M:=\bigoplus_i \Omega_iM/\Omega_{i-1}M$ is finitely generated over $\gr A$).
\end{definition}

\begin{lemma}\label{lem:good-filtration-exists}
\mbox{}
\begin{enumerate}
  \item If $M$ is finitely generated over $A$, then good filtrations on $M$ exist.
  \item If $\Omega_\bullet M$ is good, then there exist integers $k_1,\dots,k_\ell$ and
        elements $m_1,\dots,m_\ell\in M$ with $m_i\in\Omega_{k_i}M$ such that, for every $p$,
        \[
          \Omega_pM \,=\, \sum_{i=1}^{\ell} \Gamma_{p-k_i}A\cdot m_i.
        \]
\end{enumerate}
\end{lemma}
\begin{proof}[Sketch]
(1) If $M$ is generated by $m_1,\dots,m_\ell$, define
$\Omega_pM:=\sum_{i=1}^{\ell}\Gamma_pA\cdot m_i$; then $R^{\Omega}(M)$ is generated by the
homogeneous elements $m_i u^0$ over $R^{\Gamma}(A)$.
(2) Choose homogeneous generators $m_i u^{k_i}$ of $R^{\Omega}(M)$; unwinding the
definition gives the stated description of $\Omega_pM$.
\end{proof}

\begin{corollary}\label{cor:compare-good-filtrations}
Let $M$ be a finitely generated $A$-module and let $\Omega_\bullet M$ and $F_\bullet M$
be two compatible filtrations on $M$.
If $F_\bullet M$ is good, then there exists an integer $a$ such that
\[
  F_pM \subseteq \Omega_{p+a}M \qquad \text{for all } p.
\]
Consequently, if $\Omega_\bullet M$ is also good, then there exists an integer $a$ with
\[
  \Omega_{p-a}M \subseteq F_pM \subseteq \Omega_{p+a}M \qquad \text{for all } p.
\]
\end{corollary}
\begin{proof}
Assume $F_\bullet M$ is good. By Lemma~\ref{lem:good-filtration-exists}(2) there
exist integers $k_1,\dots,k_\ell$ and elements $m_i\in M$ with
$m_i\in\Omega_{k_i}M$ such that $\Omega_qM=\sum_i \Gamma_{q-k_i}A\cdot m_i$ for all $q$.
Fix $q$ and compute for any $p$:
\begin{align*}
  F_pM\;&\subseteq\; \sum_{i=1}^{\ell} \Gamma_{p-k_i}A\cdot \Omega_qM\\
  &= \sum_{i=1}^{\ell}\sum_{j=1}^{\ell} \Gamma_{p-k_i}\Gamma_{q-k_j}A\cdot m_j\\
  &\subseteq \sum_{j=1}^{\ell} \Gamma_{p+a-k_j}A\cdot m_j\\
  &= \Omega_{p+a}M,  
\end{align*}

where $a:=\max_i(k_i-q)\ge0$. This proves $F_pM\subseteq\Omega_{p+a}M$ for all $p$.
If $\Omega_\bullet M$ is also good, the same argument with the roles exchanged
gives $\Omega_{p-a}M\subseteq F_pM$ for some (possibly larger) $a$, proving the
two-sided inequality.
\end{proof}

Then for some $a\in\mathbb Z$ and all $p$ we have the chain
\[
  \Omega_{p-a}M \subseteq F_pM \subseteq \Omega_{p+a}M.
\]

\paragraph{Grothendieck group.}
For any ring $A$, let $K_0(A)$ denote the Grothendieck group of the abelian
category $\Mod(A)$: it is generated by symbols $[M]$ for $M\in\Mod(A)$ with the
relations $[M]=[M_1]+[M_2]$ for every short exact sequence
$0\to M_1\to M\to M_2\to 0$.

\section{Characteristic Varieties and Cycles of $\D_X$-modules}
Let $X=\Spec R$ be smooth affine with algebraic coordinates $(x_1,\dots,x_n)$.
Define an order function on $\D_X$ by
\[
  \ord(f)=0\ (f\in R), \qquad \ord(v)=1\ (v\in \T_X),
\]
and for a differential operator $P=\sum_{\alpha} f_{\alpha}\,\partial^{\alpha}$ set
\[
  \ord(P)=\max\{\,|\alpha|\mid f_{\alpha}\ne 0\,\},\qquad |\alpha|=\alpha_1+\cdots+\alpha_n.
\]
This gives an increasing $\mathbb Z_{\ge0}$-filtration $F_p\D_X:=\{\,P\in\D_X\mid \ord(P)\le p\,\}$.

\begin{lemma}\label{lem:grDx}
There are natural isomorphisms of $R$-algebras
\[
  \gr \D_X \cong R[\xi_1,\dots,\xi_n] \cong \Sym\,\T_X.
\]
Consequently, $\Spec(\gr\D_X)\simeq T^{\ast}X$.
\end{lemma}
\begin{proof}
One can see \cite{chen2024dmodules} or the proof below:\\
Write $F_p\D_X$ for the order filtration and $\sigma_p: F_p\D_X\twoheadrightarrow
F_p\D_X/F_{p-1}\D_X$ for the order-$p$ principal symbol. Then $\sigma_1(\partial_{x_i})=:\xi_i$
commute among themselves and with $R$, and every $\sigma_p(P)$ is a homogeneous
polynomial of degree $p$ in the $\xi_i$ with coefficients in $R$. This gives a
surjective graded $R$-algebra map $R[\xi_1,\dots,\xi_n]\twoheadrightarrow \gr\D_X$.
For injectivity, note that the kernel is a graded ideal whose degree-$1$ part is
zero (since $\sigma_1$ is injective on the span of $\partial_{x_i}$), hence it is
zero by induction on degree using the filtration and the commutation relations.
Therefore the map is an isomorphism. The second
isomorphism follows from the identification $\T_X\cong \Der_{\C}(R,R)$ and the
universal property of the symmetric algebra. Taking spectra yields
$\Spec(\gr\D_X)\cong T^{\ast}X$.
\end{proof}

\begin{corollary}\label{cor:Dx-noeth}
$\D_X$ is Noetherian.
\end{corollary}
\begin{proof}
Use Lemma~\ref{lem:grDx} and Proposition~\ref{prop:noeth}.
\end{proof}

\begin{lemma}\label{lem:gr-class-independent}
If $F_\bullet M$ and $G_\bullet M$ are two good filtrations on a finitely
generated $\D_X$-module $M$, then
\[
  [\,\gr^{F} M\,] = [\,\gr^{G} M\,] \quad \text{in } K_0(\gr\D_X).
\]
\end{lemma}
\begin{proof}
By Corollary~\ref{cor:compare-good-filtrations} there exists $a\ge0$ such that
$G_{p-a}M\subseteq F_pM\subseteq G_{p+a}M$ for all $p$. Reduce to the adjacent
case: suppose $F_{p-1}M\subseteq G_{p-1}M\subseteq F_pM\subseteq G_pM$ for all
$p$. Then we have short exact sequences
\[
  0\to G_{p-1}M/F_{p-1}M\to F_pM/F_{p-1}M\to F_pM/G_{p-1}M\to 0,
\]
\[
  0\to F_pM/G_{p-1}M\to G_pM/G_{p-1}M\to G_pM/F_pM\to 0.
\]
Taking classes in $K_0(\gr\D_X)$ and summing over $p$ yields
$[\gr^{F}M]=[\gr^{G}M]$. 

\noindent In general, $\forall k \in \mathbb{Z}$ set
\[
    F^{(k)}_i M = F_i M + G_{i+k} M \quad \forall i
\]
Then
\[
    \begin{cases}
        F^{(k)}_\bullet = F_\bullet & k \ll 0 \\
        F^{(k)}_\bullet = G_{\bullet + k} & k \gg 0
    \end{cases}
\]
and $F^{(k)}_\bullet$ and $F^{(k+1)}_\bullet$ are adjacent.

\vspace{1em} % 增加一点垂直间距

\noindent By induction, we have
\[
    [\operatorname{gr}^F_\bullet M] = [\operatorname{gr}^G_{\bullet+k} M] = [\operatorname{gr}^G_\bullet M]
\]


\end{proof}

\paragraph{Characteristic cycle and variety.}
Given a finitely generated $\D_X$-module $M$ with a good filtration $F_\bullet M$,
the associated graded $\gr^{F}M$ is a finitely generated module over
$\gr\D_X\cong R[\xi_1,\dots,\xi_n]$. Its support is a closed subset of
$T^{\ast}X\simeq\Spec(\gr\D_X)$. Let $\{\mathfrak p\}$ be the minimal primes in
$\operatorname{Supp}(\gr^{F}M)$. Define the multiplicity $\ell(\mathfrak p)$ to be the length of
the localization $(\gr^{F}M)_{\mathfrak p}$ as a module over $(\gr\D_X)_{\mathfrak p}$. The
\emph{characteristic cycle} and \emph{characteristic variety} of $M$ are
\begin{align*}
  &\mathrm{CC}(M) := \sum_{\mathfrak p\in\operatorname{Supp}(\gr^{F}M)\atop \text{minimal}}
  \ell(\mathfrak p)\,\overline{\{\mathfrak p\}},
  \qquad\\
  &\mathrm{Ch}(M) := \operatorname{Supp}\,\mathrm{CC}(M)
  = \bigcup_{\mathfrak p\,\text{minimal in }\operatorname{Supp}(\gr^{F}M)} \overline{\{\mathfrak p\}}
  \subseteq T^{\ast}X.
\end{align*}

By Lemma~\ref{lem:gr-class-independent}, $\mathrm{CC}(M)$ and $\mathrm{Ch}(M)$ do not depend on the
choice of good filtration.












