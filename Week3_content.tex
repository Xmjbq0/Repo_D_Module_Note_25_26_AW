\providecommand{\C}{\mathbb{C}}
\providecommand{\A}{\mathbb{A}}
\providecommand{\Z}{\mathbb{Z}}
\providecommand{\Spec}{\operatorname{Spec}}
\providecommand{\Supp}{\operatorname{Supp}}
\providecommand{\Ch}{\operatorname{Ch}}
\providecommand{\gr}{\operatorname{gr}}
\providecommand{\Ann}{\operatorname{Ann}}
\providecommand{\rad}{\operatorname{rad}}
\providecommand{\Hom}{\operatorname{Hom}}
\providecommand{\End}{\operatorname{End}}
\providecommand{\D}{\mathcal{D}}
\providecommand{\T}{\mathcal{T}}
\providecommand{\Poisson}{\mathrm{P}}
\providecommand{\Rees}{\operatorname{R}}

% 定义切空间符号,手写体通常是 \mathcal{T}
\newcommand{\Tv}{\mathcal{T}}
% 为了美观,定义 smooth 的上标为正体
\newcommand{\sm}{\mathrm{sm}}

\section{Symplectic vector spaces}

Let $V$ be a $k$-vector space equipped with a bilinear form
$\sigma\colon V\times V\to k$.  We use the following terminology:
\begin{itemize}
  \item The form $\sigma$ is \emph{skew-symmetric} if
        $\sigma(a,b)=-\sigma(b,a)$ for all $a,b\in V$.
  \item A skew-symmetric bilinear form is \emph{nondegenerate} if for every
        $0\ne a\in V$ there exists $v\in V$ such that $\sigma(a,v)\ne 0$.
\end{itemize}
A pair $(V,\sigma)$ is called a \emph{symplectic vector space} if $\sigma$
is a nondegenerate skew-symmetric bilinear form.  In this case
$\dim_k V=2n$ for some $n$.

Nondegeneracy identifies $V$ with its dual: the map
\[
  \sharp \colon V \longrightarrow V^\ast,\qquad
  v \longmapsto \sigma(-,v)
\]
is an isomorphism, called the \emph{Hamiltonian isomorphism}.

\begin{definition}[Def.~7.1]
Let $W\subseteq V$ be a linear subspace.
\begin{enumerate}
  \item The \emph{symplectic orthogonal} of $W$ is
        $W^\perp:=\{v\in V\mid \sigma(v,W)=0\}$.
  \item The subspace $W$ is called \emph{isotropic} if $W\subseteq W^\perp$,
        \emph{Lagrangian} if $W=W^\perp$, and \emph{coisotropic} if
        $W\supseteq W^\perp$.
\end{enumerate}
\end{definition}

\begin{lemma}[Lemma~7.2]
Let $W\subseteq V$ be as above.  If $W$ is isotropic then $\dim_k W\le n$.
If $W$ is Lagrangian then $\dim_k W=n$, and if $W$ is coisotropic then
$\dim_k W\ge n$.
\end{lemma}
\begin{proof}
The Hamiltonian isomorphism identifies $W^\perp$ with the kernel of the
composition $V\xrightarrow{\sharp}V^\ast\twoheadrightarrow W^\ast$.  Hence
$\dim_k W^\perp=2n-\dim_k W$.  When $W\subseteq W^\perp$ we obtain
$\dim_k W\le n$.  Equality occurs exactly when $W=W^\perp$, the defining
property of a Lagrangian subspace.  Dually, $W\supseteq W^\perp$ implies
$\dim_k W\ge n$, which is the coisotropic case.
\end{proof}

\begin{example}[Ex.~7.2]\label{ex:symplectic-standard}
Let $W$ be an $n$-dimensional $k$-vector space with dual $W^\ast$.  Then
$V:=W\oplus W^\ast$ becomes a symplectic vector space when we define
\[
  \sigma\big((x,\xi),(x',\xi')\big)
  :=\langle x',\xi\rangle-\langle x,\xi'\rangle,
  \qquad (x,\xi),(x',\xi')\in V,
\]
where $\langle\cdot,\cdot\rangle$ denotes the evaluation pairing between
$W$ and $W^\ast$.
\end{example}

\section{Symplectic structure on $T^\ast X$}

Let $X=\Spec R$ be a smooth affine algebraic variety over $\C$ (more generally
over an algebraically closed field $k$ of characteristic $0$) with algebraic
coordinates $(x_1,\dots,x_n)$.  The cotangent bundle is
\[
  T^\ast X = \Spec \gr \D_X = \Spec \Sym^\bullet T_X
           \simeq \Spec R[\xi_1,\dots,\xi_n],
\]
where $\Sym^\bullet T_X:=\bigoplus_{i\ge0}\Sym^i T_X$ carries the natural
grading.

\begin{definition}[Def.~8.1]
A subvariety $V\subseteq T^\ast X$ is called \emph{conic} if its defining ideal
$I(V)\subseteq \Sym^\bullet T_X$ is homogeneous.
\end{definition}

\begin{remark}
Under the identification $T^\ast X\simeq X\times k^n$, each fiber is an
$n$-dimensional $k$-vector space, so $T^\ast X$ carries a canonical $k^\times$
action.  A subvariety $V\subseteq T^\ast X$ is conic precisely when it is
stable under this action.
\end{remark}

The canonical $1$-form on the cotangent bundle is
\[
  \alpha_X := \sum_{i=1}^n \xi_i\,dx_i \in \Omega^1_{T^\ast X},
\]
which does not depend on the chosen coordinates.  Its exterior derivative
\[
  \sigma_X := d\alpha_X = \sum_{i=1}^n d\xi_i\wedge dx_i \in \Omega^2_{T^\ast X}
\]
is the standard symplectic form on $T^\ast X$.

For a closed point $p\in T^\ast X$ with maximal ideal $\mathfrak m_p$ we have
the tangent space
\[
  T_{T^\ast X,p}=\Der_k(\mathfrak m_p,k)
               \simeq \Omega^1_{T^\ast X}\otimes_S S/\mathfrak m_p
               \simeq \mathfrak m_p/\mathfrak m_p^2.
\]
Evaluating $\sigma_X$ at $p$ yields a bilinear form
$\sigma_p\colon T_{T^\ast X,p}\times T_{T^\ast X,p}\to k$.

\begin{lemma}[Lemma~8.1]
For every closed point $p\in T^\ast X$, the pair $(T_{T^\ast X,p},\sigma_p)$ is
a symplectic vector space.
\end{lemma}
\begin{proof}
The identification above shows that $T_{T^\ast X,p}\simeq
\Omega^1_{T^\ast X}\otimes_S k(p)$.  Example~\ref{ex:symplectic-standard}
(with $V=\Omega^1_{T^\ast X}\otimes k(p)$ and $\sigma=\sigma_p$) applies
directly, so $\sigma_p$ is a nondegenerate skew-symmetric form.
\end{proof}

The global form $\sigma_X$ yields the Hamiltonian isomorphism
\[
  \mathcal H\colon \Omega^1_{T^\ast X}\xrightarrow{\sim}T_{T^\ast X},\qquad
  \beta\longmapsto \mathcal H(\beta),
\]
characterized by $\sigma_X(-,\mathcal H(\beta))=\beta$.  For any function
$f\in \Sym^\bullet T_X$ we define its \emph{Hamiltonian vector field} to be
\[
  H_f:=\mathcal H(df)\in \Gamma(T^\ast X,T_{T^\ast X}),
\]
which satisfies $\sigma_X(v,H_f)=df(v)$ for all vector fields $v$.

\begin{lemma}[Lemma~8.2]\label{lem:hamiltonian-coordinates}
With respect to the coordinates $(x_1,\dots,x_n,\xi_1,\dots,\xi_n)$ one has
\[
  H_f = \sum_{i=1}^n
        \left(\frac{\partial f}{\partial \xi_i}\frac{\partial}{\partial x_i}
             -\frac{\partial f}{\partial x_i}\frac{\partial}{\partial \xi_i}
        \right).
\]
\end{lemma}
\begin{proof}
Write $H_f=\sum_i \big(\alpha_i\,\partial_{x_i}+\beta_i\,\partial_{\xi_i}\big)$.
Because $\sigma_X(\partial_{x_i},\partial_{x_j})=0$ and
$\sigma_X(\partial_{x_i},\partial_{\xi_j})=\delta_{ij}$, we obtain
\[
  df(\partial_{x_i})=\sigma_X(\partial_{x_i},H_f)=-\beta_i,\qquad
  df(\partial_{\xi_i})=\sigma_X(\partial_{\xi_i},H_f)=\alpha_i.
\]
Thus $\alpha_i=\partial f/\partial \xi_i$ and $\beta_i=-\partial f/\partial
x_i$, yielding the displayed formula.
\end{proof}

The Hamiltonian vector fields endow $\Sym^\bullet T_X$ with a Poisson bracket:
for $f,g\in \Sym^\bullet T_X$ set
\[
  \{f,g\}_\sigma := H_f(g)=\sigma_X(H_f,H_g).
\]
It is skew-symmetric, satisfies the Jacobi identity, and acts as a derivation
in each variable.

\begin{lemma}[Lemma~8.3]
$(\Sym^\bullet T_X,\{-,-\}_\sigma)$ is a Poisson algebra.
\end{lemma}
\begin{proof}
The identities above translate the usual properties of Hamiltonian vector
fields into the axioms of a Poisson bracket; this is standard and was assigned
as homework.
\end{proof}

\begin{lemma}[Lemma~8.4]
Under the identification $\gr \D_X\simeq \Sym^\bullet T_X$, the Poisson bracket
constructed above coincides with $\{-,-\}_\sigma$.
\end{lemma}
\begin{proof}
Because $\gr \D_X$ is a graded $R$-algebra isomorphic to $\Sym^\bullet T_X$, it
is enough to compare the brackets on elements of the form
$f=f_\alpha\xi^\alpha$, $g=g_\beta\xi^\beta$ with $f_\alpha,g_\beta\in R$ and
multi-indices $\alpha,\beta\in\mathbb Z_{\ge0}^n$.  Using
Lemma~\ref{lem:hamiltonian-coordinates} we compute
\[
  \{f,g\}_\sigma
   = \sum_{i=1}^n
       \left(
         \alpha_i f_\alpha\,\frac{\partial g_\beta}{\partial x_i}
         -\beta_i g_\beta\,\frac{\partial f_\alpha}{\partial x_i}
       \right)
       \xi^{\alpha+\beta-e_i},
\]
where $e_i$ is the $i$th standard basis vector.  This matches the explicit
formula obtained earlier for the bracket on $\gr \D_X$, so the two Poisson
structures agree.
\end{proof}

Now let $S:=\Sym^\bullet T_X$, so $T^\ast X=\Spec S$.  Let
$V\subseteq T^\ast X$ be a closed subvariety with radical ideal
$I_V\subseteq S$.  For a smooth closed point $p\in V$ write
$\mathfrak m_p\subset S$ and $\overline{\mathfrak m}_p\subset S/I_V$ for the
corresponding maximal ideals.  The exact sequence
\[
  0 \longrightarrow I_{V,p}/I_{V,p}^2
    \longrightarrow \mathfrak m_p/\mathfrak m_p^2
    \longrightarrow \overline{\mathfrak m}_p/\overline{\mathfrak m}_p^2
    \longrightarrow 0
\]
and we denote $\mathfrak m_p/\mathfrak m_p^2 \longrightarrow \overline{\mathfrak m}_p/\overline{\mathfrak m}_p^2$
in above short exact sequence by \hypertarget{triple_ast}{$\textcolor{red}{(* * *)}$},
and this short exact sequence dualizes to a natural embedding of tangent spaces
\[
  T_{V,p}\simeq (\overline{\mathfrak m}_p/\overline{\mathfrak m}_p^2)^\vee
   \hookrightarrow
  T_{T^\ast X,p}\simeq (\mathfrak m_p/\mathfrak m_p^2)^\vee.
\]
Equivalently, $T_{V,p}$ consists of those vectors $v\in T_{T^\ast X,p}$ such
that $df(v)=0$ for every $f\in I_V$.

\begin{definition}[Def.~8.4]
A closed subvariety $V\subseteq T^\ast X$ is \emph{isotropic} (resp.
\emph{Lagrangian}, \emph{coisotropic}) if for every smooth closed point
$p\in V$ the tangent space $T_{V,p}$ is isotropic (resp. Lagrangian,
coisotropic) inside the symplectic vector space $T_{T^\ast X,p}$.
\end{definition}

\begin{theorem}[Thm.~8.5]
Let $V\subseteq T^\ast X$ be a closed subvariety with ideal $I_V$.  The
following are equivalent:
\begin{enumerate}
  \item $V$ is coisotropic.
  \item $I_V$ is involutive, i.e.\ $\{I_V,I_V\}_\sigma\subseteq I_V$.
\end{enumerate}
\end{theorem}
\begin{proof}
Take a smooth closed point $p \in V$.
\begin{gather*}
    \sigma(v, H_f) = df(v) \quad \text{and} \quad \hyperlink{triple_ast}{\textcolor{red}{(* * *)}} \\
    \Downarrow \\ % 换行直接画箭头
    \begin{align*}
      (\Tv_{V,p})^\perp &= \{ v \in \Tv_{T^*X, p} \mid \sigma_p(v, w) = 0, \quad \forall w \in \Tv_{V,p} \} \\
      &= \text{the space spanned by } H_{f,p} \quad \forall f \in I_V.
    \end{align*}
\end{gather*}

% 中间的方框注释
\noindent If $h \in S$, then $h_p$ denotes the value of $h$ at the point $p$, i.e. $h_p = \bar{h} \in \frac{R}{\mathfrak{m}_p} \simeq k$.


\vspace{1em}

\noindent $V$ coisotropic
\begin{align*}
    &\implies H_{f,p} \in \Tv_{V,p} \implies \{f, g\}_p \\
    &= H_f(g)_p = dg(H_f)_p = 0, \quad \forall \text{ regular } p.
\end{align*}
\[
    \implies \exists \text{ open } U \subseteq V \quad \text{s.t.} \quad \forall p \in U \text{ is regular.}
\]
\[
    \{f, g\}|_U = 0 \xrightarrow{HW} \{f, g\}|_V = 0.
\]
So (1) $\Rightarrow$ (2).

\vspace{1em}
\noindent (2) $\Rightarrow$ (1) is similar.
\end{proof}

As an immediate corollary we obtain the standard statement for characteristic
varieties.

\begin{theorem}[Thm.~8.6]
Let $\mathcal M$ be a finitely generated $\D_X$-module.  Then its
characteristic variety $\Ch(\mathcal M)\subseteq T^\ast X$ is conic and
coisotropic.  In particular $\dim \Ch(\mathcal M)\ge n$.
\end{theorem}

\section{Conic Lagrangian subvarieties of $T^\ast X$}

Write $Y=\Spec S$ for a smooth affine $k$-variety, and let
$V\subseteq Y$ be a closed subscheme with radical ideal $I=I_V\subseteq S$.
Set $\overline S:=S/I$, so $V=\Spec \overline S$.

\begin{lemma}[Lemma~9.1]\label{lem:kahler-sequence}
There is an exact sequence of $\overline S$-modules
\[
  I/I^2 \xrightarrow{\;\delta\;} \Omega_S\otimes_S \overline S
    \longrightarrow \Omega_{\overline S}\longrightarrow 0,\qquad
  \delta(\overline b)=db\otimes 1.
\]
\end{lemma}
\begin{proof}
The Leibniz rule shows that $\delta$ is well-defined: $d(bb')=b\,db'+b'\,db$.
By the universal property of Kähler differentials the cokernel of $\delta$ is
$\Omega_{\overline S}$, yielding the claimed exact sequence.
\end{proof}

Hence $\Omega_{\overline S}$ is finitely generated over $\overline S$.

\begin{definition}[Def.~9.2]
Let $V=\Spec \overline S$ be as above.
\begin{enumerate}
  \item The \emph{smooth locus} of $V$ is
        \[
          V^{\mathrm{sm}}
          :=\{p\in \Spec \overline S\mid
                (\Omega_{\overline S})_p
                \text{ is free over }\overline S_p\}.
        \]
        For an affine algebraic variety the smooth locus is a non-empty open
        subset (homework).
  \item A closed subscheme $V\subseteq Y$ is a \emph{smooth subscheme} if
        $V=V^{\mathrm{sm}}$.
\end{enumerate}
\end{definition}

\begin{proposition}[Prop.~9.2]
Let $V\subseteq Y$ be an irreducible subscheme.  The following are equivalent:
\begin{enumerate}
  \item $V\subseteq Y$ is a smooth subscheme.
  \item $\Omega_{\overline S}$ is locally free.
\end{enumerate}
Moreover, in this case the sequence of Lemma~\ref{lem:kahler-sequence}
restricts to a short exact sequence
\[
  0\longrightarrow I/I^2
    \longrightarrow \Omega_Y\otimes_S \overline S
    \longrightarrow \Omega_{\overline S}
    \longrightarrow 0,
\]
so $I/I^2$ is locally free of rank $\operatorname{codim}_Y V$.
\end{proposition}

\begin{proof}
Only need to prove ``$\Leftarrow$'':
    
\noindent By Lemma 9.1, we have
\[
    I/I^2 \to \Omega_Y \otimes_S \bar{S} \xrightarrow{\varphi} \Omega_{\bar{S}} \to 0
\]
Since $\Omega_{\bar{S}}$ is locally free, $\ker \varphi$ is also locally free.

\noindent Take $U' \subseteq Y$, an open affine neighborhood s.t. $U = U' \cap V$ and $\ker \varphi|_U$ is free of rank $r = \dim Y - \dim V$ with generators $dx_1, \dots, dx_r \in \Omega_Y$.

\noindent Lift $dx_i$ to $x_i \in I|_{U'}$.

\noindent Take $I' = \langle x_1, \dots, x_r \rangle$ and
\[
    V' = \Spec \bar{S}', \quad \bar{S}' = S/I'
\]
Thanks to Lemma 9.1 again, we have
\[
    I'/{I'}^2 \xrightarrow{\sigma'} \Omega_Y \otimes_S S' \xrightarrow{\varphi'} \Omega_{S'} \to 0
\]
So $\ker \varphi'$ is generated by $dx_1, \dots, dx_r$ and thus also free of rank $r$.

\noindent Since $I'$ is generated by $x_1, \dots, x_r$ and $\Img(\sigma')$ is free, $\sigma'$ is injective.

\noindent So $\Omega_{S'}$ is free $\implies V' \subseteq Y$ smooth.

\noindent So $V \subseteq V'$ is a smooth subvariety of the same dimension.

\noindent So locally, $V = V'$ and $\sigma = \sigma'$.

\noindent So $\sigma$ is injective.
\end{proof}

\noindent Now, we assume
\[
    Z \subset X \quad \text{smooth irreducible subvar.}
\]
\[
    I \subseteq R, \text{ the ideal of } Z
\]
then $I/I^2$ is a locally free $\bar{R} = R/I$-mod.
\\
and we have a s.e.s. (short exact sequence)
\[
    0 \to \Tv_Z \to \underset{:= \Tv_X \otimes_R \bar{R}}{\Tv_X|_Z} \to (I/I^2)^* \to 0
\]

\noindent For $v \in \Tv_{X,z}$, we say $v$ is tangent to $Z$ at $z \in Z$
\[
    \text{if } v \in \Tv_{Z,z} \quad (\iff v|_{(I/I^2)_z} = 0)
\]
We define
\begin{align*}
    T^*_Z X &= \Spec \Sym (I/I^2)^* \\
    &\hookrightarrow \underset{\substack{\mid \mid \\ \Spec \bar{R}[\xi_1, \dots, \xi_n]}}{\Spec \Sym \Tv_X|_U} \hookrightarrow \underset{\substack{\mid \mid \\ T^*X}}{\Spec \Sym^\bullet \Tv_X}
\end{align*}
called the \emph{conormal bundle} of $Z \subset X$.

Over an affine open $U\subseteq X$
where $I/I^2$ is free we can choose coordinates $(x_1,\dots,x_n)$ with
$I=(x_1,\dots,x_r)$. \\
Then $T^\ast_Z X$ carries coordinates
$(\xi_1,\dots,\xi_r,x_{r+1},\dots,x_n)$.

If $Z\subseteq X$ is irreducible, we define its conormal variety by
\[
  \Lambda_Z := \overline{T^\ast_{Z^{\mathrm{sm}}} X}
  \subseteq T^\ast X,
\]
the closure of the conormal bundle over the smooth locus.  On a suitable open
subset $U\subseteq X$ with $Z\cap U=Z^{\mathrm{sm}}\cap U$ this agrees with the
previous description.

\begin{theorem}[Thm.~9.3]\label{thm:conic-lagrangian}
Let $\Lambda\subseteq T^\ast X$ be an irreducible conic Lagrangian subvariety.
Then $\Lambda=\Lambda_Z$ for some irreducible subvariety $Z\subseteq X$.
\end{theorem}

To analyze conic subvarieties we use the Euler vector field.  Recall that the
canonical one-form is $\alpha_X=\sum_i \xi_i\,dx_i$.  Its Hamiltonian
\[
  H(\alpha_X)=\sum_i \alpha_i\partial_{x_i}+\beta_i\partial_{\xi_i}
\]
is determined by $\sigma_X(\partial_{x_i},H(\alpha_X))=\alpha_X(\partial_{x_i})
=\xi_i$ and $\sigma_X(\partial_{\xi_i},H(\alpha_X))=\alpha_X(\partial_{\xi_i})
=0$.  Hence $H(\alpha_X)=-\sum_i \xi_i\,\partial_{\xi_i}$, the negative of the
Euler vector field, and its flow generates the $k^\times$-action on $T^\ast X$.

% \begin{lemma}[Lemma~9.4]\label{lem:tangent_vector}
\begin{lemma}[Lemma~9.4]\label{lem:tangent_vector}
If $V\subseteq T^\ast X$ is conic with ideal $I_V$, then $H(\alpha_X)$ is
tangent to $V$, i.e.\ $H(\alpha_X)(I_V)\subseteq I_V$.
\end{lemma}
\begin{proof}
Because $I_V$ is homogeneous we may assume its generators are of the form
$f_\alpha\,\xi^\alpha$ with $f_\alpha\in R$.  The Euler vector field acts by
$H(\alpha_X)(f_\alpha\,\xi^\alpha)=-|\alpha|\,f_\alpha\,\xi^\alpha$, which lies
again in $I_V$.
\end{proof}

\begin{lemma}[Lemma~9.5]\label{lem:9.5}
Let $V\subseteq T^\ast X$ be a smooth conic subvariety.  Then $V$ is isotropic
if and only if $\alpha_X|_V\equiv 0$.
\end{lemma}
\begin{proof}
\noindent ``$\Leftarrow$'': 
\[
    \alpha_X|_V \equiv 0 \implies \sigma_X|_V \equiv 0 \implies V \text{ is isotropic.}
\]

\noindent ``$\Rightarrow$'': 
Take $v \in \Tv_V$.

\vspace{1em}
\noindent By Lemma~\ref{lem:tangent_vector}, $H(\alpha_X)|_V$ is tangent to $V$.
\[
    V \text{ isotropic} \implies \text{tangent vectors of } V \text{ at } p \subseteq \Tv_{V,p}^\perp
\]
So
\[
    \langle \alpha_X|_V, v \rangle = \sigma_X|_V ( v, H(\alpha_X|_V) ) = 0
\]
So
\[
    \alpha_X|_V \equiv 0. \qedhere
\]
\end{proof}

\begin{corollary}[Cor.~9.6]
Let $Z\subseteq X$ be irreducible.  Then $T^\ast_Z X\subseteq T^\ast X$ is a
conic Lagrangian subvariety.
\end{corollary}
\begin{proof}
Taking loc.\ coor.\ $(x_1, \dots, x_n, \xi_1, \dots, \xi_n)$
s.t.\ $T^*_{Z^{\sm}} X$ has coor.
\[
    (\xi_1, \dots, \xi_r, x_{r+1}, \dots, x_n)
\]
So
\[
    \alpha_X|_{T^*_{Z^{\sm}} X} \equiv 0.
\]
\[
    \implies T^*_{Z^{\sm}} X \text{ is isotropic}
\]
and since
\[
    \dim T^*_{Z^{\sm}} X = n
\]
\[
    \implies T^*_{Z^{\sm}} X \text{ Lagrangian}
\]
$T^*_{Z^{\sm}} X$ conic $\implies \overline{T^*_{Z^{\sm}} X}$ conic. \qedhere
\end{proof}

\begin{proof}[Proof of Theorem~\ref{thm:conic-lagrangian}]
Let $\pi\colon T^\ast X\to X$ denote the projection and set $Z:=\pi(\Lambda)$.
Because $\Lambda$ is conic, its image is closed and irreducible; write
$\dim Z=n-r$.  The smooth locus $Z^{\mathrm{sm}}$ lifts to the open subset
$\Lambda_0:=\pi^{-1}(Z^{\mathrm{sm}})\cap \Lambda\subseteq \Lambda$.  Choosing
coordinates
\[
  (x_1,\dots,x_n,\xi_1,\dots,\xi_n)
\]
around a closed point $z\in Z^{\mathrm{sm}}$ with
$Z^{\mathrm{sm}}=\{x_1=\cdots=x_r=0\}$ and
$T^\ast_{Z^{\mathrm{sm}}} X=\{\xi_1=\cdots=\xi_r=\xi_{r+1}=\cdots=\xi_n=0\}$,
\[
    \alpha_X|_{\Lambda_0} = \xi_{r+1} dx_{r+1} + \dots + \xi_n dx_n
\]
By Lemma~\ref{lem:9.5}, $\forall p \in \Lambda_0$, $\xi_{i,p} = 0 \quad \forall i = r+1, \dots, n$.
\[
    \implies \Lambda_0 \subseteq T^*_{Z^{\sm}} X
\]
Since $\dim \Lambda_0 = n$, $\Lambda = \overline{\Lambda_0} = \overline{T^*_{Z^{\sm}} X} = T^*_Z X$.
\end{proof}

\begin{definition}[Def.~9.7]
Let $\mathcal M$ be a finitely generated $\D_X$-module.  We say that
$\mathcal M$ is \emph{holonomic} if $\dim \Ch(\mathcal M)=n$.
\end{definition}

\begin{corollary}
If $\mathcal M$ is holonomic, then its characteristic cycle decomposes as
\[
  \operatorname{CC}(\mathcal M)
    = \sum_i \ell_i\, T^\ast_{Z_i} X,
\]
where each $Z_i\subseteq X$ is an irreducible subvariety.
\end{corollary}
